\documentclass[12pt, a4paper]{ctexart} % 默认小四号字

%todo 页面设置  
\usepackage[a4paper,left=3.18cm,right=3.18cm,top=2.54cm,bottom=2.54cm]{geometry}



% 默认字体字号
\usepackage{fontspec}
\setmainfont{Times New Roman}  % 英文是Times New Roman
\usepackage{xeCJK}
\setCJKmainfont{SimSun}[BoldFont= SimHei, ItalicFont = KaiTi ] % 中文默认字体:宋体,粗体以黑体代替,斜体以楷书代替



% todo 目录
% 目录字体设置
\usepackage{subfigure}
\usepackage[subfigure]{tocloft}
\renewcommand*\contentsname{\centerline{ \heiti \zihao{-2} 目录 }} % 目录设置
\setcounter{secnumdepth}{3} % 编号深度
\setcounter{tocdepth}{3} % 目录深度
\renewcommand{\cftsecfont}{\songti} % 目录内容字体
\renewcommand{\cftsecpagefont}{\songti} % 页码字体

%-------------------章节与段落---------------------------
%\setlength{\parskip}{1\baselineskip}  % 段间距,基本间距的倍数
%\ctexset{section={format={\heiti \zihao{-2} }, beforeskip=12pt,afterskip=12pt}, % 多级标题格式,可加/centering
%subsection={format={\heiti \zihao{4}},beforeskip=12pt,afterskip=0pt},          % 例:一级标题
%subsubsection={format={\heiti \zihao{-4}},beforeskip=12pt,afterskip=0pt}}	   % 黑体,小二号,段前12磅,段后12磅

%------------------图表公式代码-------------------------
\usepackage{graphicx} % 图片
\graphicspath{{pic/}{logo/}}  % pic为图片存放路径
\usepackage{caption} % 图表名字体字号
\captionsetup{font={small,bf},labelfont={small,bf}} % 比标准字号小一号,加粗
\usepackage{amsmath}
\numberwithin{equation}{section}%公式按章节编号
\numberwithin{figure}{section}%图表按章节编号
\usepackage{float}%提供float浮动环境
\usepackage{booktabs} % 三线表,提供命令\toprule、\midrule、\bottomrule
\usepackage{algorithmic} % 伪代码
\usepackage{algorithm}
\usepackage{fancyhdr}

% todo 标题

\title{\vspace{-3cm} \heiti \zihao{5} \LARGE 论文标题}%黑体2号加粗
%\author{\kaishu \zihao{-4} 小明\\\songti \zihao{-5} 加里敦大学}%作者姓名楷体小四,学院地址宋体小五
\date{}
%--------------------摘要----------------------------
%\newenvironment{enabstract}{
	% \noindent 
	%}

% todo 标题
\newcommand\thesistitle[1]{\begin{center}
		 \zihao{5} \LARGE \textbf{#1}
	\end{center}\par\par}

\newcommand\thesisentitle[1]{\begin{center}
	\zihao{-2} \textbf{#1}
	\end{center}\par\par}

% todo 摘要
\newenvironment{cnabstract}{
	\noindent \textbf{摘要:} 
}

\newenvironment{enabstract}{
	\noindent \textbf{Abstract:} 
}

\usepackage{color,xcolor}
\pagestyle{fancy}
%\renewcommand{\headrule}{\color{gray}\xhrulefill[height=-2pt,thickness=1pt]}
\lhead{} 
\chead{\footnotesize \yahei \textcolor{gray}{基于DenseNet和注意力机制的数学公式识别}} 
\rhead{\footnotesize \yahei \textcolor{gray}{\thepage}}
\lfoot{} \cfoot{} \rfoot{}
\renewcommand{\headrulewidth}{0.2pt} 
\setlength{\headwidth}{\textwidth}



%%%%%%%%%%%%%%%%%%%%%%%%%%%%%导言/正文分割线%%%%%%%%%%%%%%%%%%%%
\usepackage{setspace}
\usepackage{hyperref} % 超链接
\usepackage{cite} % 引用
\begin{document}


\begin{spacing}{1.3541667}%段落行距设置
	\zihao{4}
\section{第一章}
\subsection{二级标题}
\subsubsection{三级标题}
凑字数凑字数凑字数凑字数凑字数凑字数凑字数凑字数凑字数凑字数凑字数凑字数凑字数凑字数凑字数凑字数凑字数凑字数凑字数凑字数凑字数凑字数凑字数凑字数凑字数凑字数凑字数凑字数凑字数凑字数凑字数凑字数凑字数凑字数凑字数凑字数凑字数凑字数凑字数凑字数凑字数凑字数凑字数凑字数凑字数凑字数凑字数凑字数凑字数凑字数凑字数凑字数凑字数凑字数凑字数凑字数凑字数凑字数凑字数凑字数凑字数凑字数凑字数凑字数凑字数凑字数凑字数凑字数凑字数凑字数凑字数凑字数\textsuperscript{\cite{ref1}}。正文正文正文正文正文\par

%经典三线表
\begin{table}[H]
    \caption{\textbf{Example 1}}%标题
    \centering%把表居中
    \begin{tabular}{cccc}%四个c代表该表一共四列,内容全部居中
    \toprule%第一道横线
    Item 1&Item 2&Item 3&Item 4 \\
    \midrule%第二道横线 
    Data1&Data2&Data3&Data4 \\
    Data5&Data6&Data7&Data8 \\
    \bottomrule%第三道横线
    \end{tabular}
\end{table}
% 伪代码
\begin{algorithm}[!h]
    \caption{algorithm of SUM}
    \label{alg:AOA}
    \renewcommand{\algorithmicrequire}{\textbf{Input:}}
    \renewcommand{\algorithmicensure}{\textbf{Output:}}
    \begin{algorithmic}[1]
        \REQUIRE $A$, $B$, $C$  %%input
        \ENSURE EEEEE    %%output
        
        \STATE  AAAAA
        \WHILE{$A=B$}
            \STATE BBBBB
        \ENDWHILE
        
        \FOR{each $i \in [1,10]$}
            \IF {$C = 0$}
                \STATE CCCCC
            \ELSE
                \STATE DDDDD
            \ENDIF
        \ENDFOR
        
        \RETURN EEEEE
    \end{algorithmic}
\end{algorithm}
\section{第二章}
\subsection{二级标题}
\subsubsection{三级标题}
凑字数凑字数凑字数凑字数凑字数凑字数凑字数凑字数凑字数凑字数凑字数凑字数凑字数凑字数凑字数凑字数凑字数凑字数凑字数凑字数凑字数凑字数凑字数凑字数凑字数凑字数凑字数凑字数凑字数凑字数凑字数凑字数凑字数凑字数凑字数凑字数凑字数凑字数凑字数凑字数凑字数凑字数凑字数凑字数凑字数凑字数凑字数凑字数凑字数凑\par
% 公式
\begin{equation}
    \mathrm{MCD}(\boldsymbol{y},\boldsymbol{\hat{y}})=\frac{10\sqrt{2}}{\ln10}||\boldsymbol{y}-\boldsymbol{\hat{y}}||_2
\end{equation}
% 图片
%\begin{figure}[h]
%    \centering
%    \includegraphics[width=1.0\textwidth]{1.png}
%    \caption{语音转换的基本流程}
%    \label{2.1}
%\end{figure} 
凑字数凑字数凑字数凑字数凑字数凑字数凑字数凑字数凑字数凑字数凑字数凑字数凑字数凑字数凑字数凑字数凑字数凑字数凑字数凑字数凑字数凑字数凑字数凑字数凑字数凑字数凑字数凑字数凑字数凑字数凑字数凑字数凑字数凑字数凑字数凑字数凑字数凑字数凑字数凑字数凑字数凑字数凑字数凑字数凑字数凑字数凑字数凑字数凑字数凑字数凑字数凑字数凑字数\par
\subsection{xxxx}
\subsubsection{xxx}
凑字数凑字数凑字数凑字数凑字数凑字数凑字数凑字数凑字数凑字数凑字数凑字数凑字数凑字数凑字数凑字数凑字数凑字数凑字数凑字数凑字数凑字数凑字数凑字数凑字数凑字数凑字数凑字数凑字数凑字数凑字数凑字数凑字数凑字数凑字数凑字数凑字数凑字数凑字数凑字数凑字数凑字数凑字数凑字数凑字数凑字数凑字数凑字数凑字数凑字数凑字数凑字数凑字数凑字数凑字数凑字数凑字数凑字数凑字数凑
\end{spacing}

\end{document}